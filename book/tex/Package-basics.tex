\chapter{Package basics}

An R package is the basic unit of reusable code. You need to master the
art of making R packages if you want others to use your code. At their
heart, packages are quite simple and only have two essential components:

\begin{itemize}
\item
  a \texttt{DESCRIPTION} file that describes the package, what it does,
  who's allowed to use it (the license) and who to contact if you need
  help
\item
  an \texttt{R} directory that contains your R code.
\end{itemize}

If you want to distribute R code to someone else, there's no excuse not
to use a simple package: it's a standard structure, and you can easily
expand on it by adding documentation, data and tests.

This document explains how to get started, with a description of package
structure, tips for naming your package, and more details about the
\texttt{DESCRIPTION} file.

The most accurate resource for up-to-date details on package development
is always the official
\href{http://cran.r-project.org/doc/manuals/R-exts.html\#Creating-R-packages}{writing
R extensions} guide. However, it's rather hard to read and follow if
you're not already familiar with the basics of packages. It's also
exhaustive, covering every possible package component, rather than
focussing on the most common and useful components as this package does.
Once you are familiar with the content here, you should find R
extensions a little easier to read.

\section{Package essentials}

As mentioned above, there are only two elements that you must have:

\begin{itemize}
\item
  the \texttt{DESCRIPTION} file, which provides metadata about the
  package, and is described in the following section.
\item
  the \texttt{R/} directory where your R code lives (in \texttt{.R} or
  \texttt{.r} files).
\end{itemize}

If you don't want to create this by hand, you can use
\texttt{devtools::create} which initialises the directory structure and
includes a few other files that most packages have.

\section{Optional components}

Almost all R packages also have:

\begin{itemize}
\itemsep1pt\parskip0pt\parsep0pt
\item
  the \texttt{man/} directory where your {[}{[}function
  documentation\textbar{}documenting-functions{]}{]}. In the style of
  package development described in this book, you'll never personally
  touch the files in this directory. Instead, they will be automatically
  generated from comments in your source code using the
  \texttt{roxygen2} package
\end{itemize}

After the code and function documentation, the most important optional
components of an R package help your users learn how to use your
package. The following files and directories are described in more
detail in {[}{[}documenting packages{]}{]}.

\begin{itemize}
\item
  the \texttt{NEWS} file describes the changes in each version of the
  package. Using the standard R format will allow you to take advantage
  of many automated tools for displaying changes between versions.
\item
  the \texttt{README} file gives a general overview of your package,
  including why it's important. This text should be included in any
  package announcement, to help others understand why they might want to
  use your package.
\item
  the \texttt{inst/CITATION} file describes how to cite your package. If
  you have published a peer reviewed article which you'd like people to
  cite when they use your software, this is the place to put it.
\item
  the \texttt{demo/} directory contains larger scale demos, that use
  many features of the package.
\item
  the \texttt{vignettes/} directory is used for larger scale
  documentation, like vignettes, long-form documents which show how to
  combine multiple parts of your package to solve problems.
\end{itemize}

Other optional files and directories are part of good development
practice:

\begin{itemize}
\item
  a \texttt{NAMESPACE} file describes which functions are part of the
  formal API of the package and are available for others to use. See
  {[}{[}namespaces{]}{]} for more details.
\item
  \texttt{tests/} and \texttt{inst/tests/} contains {[}{[}unit
  tests\textbar{}testing{]}{]} which ensure that your package is
  operating as designed. In this book, we'll focus on the
  \texttt{testthat} package for writing tests.
\item
  the \texttt{data/} directory contains \texttt{.rdata} files, used to
  include sample datasets (or other R objects) with your package.
\end{itemize}

There are other directories that we won't cover. You might see these in
other packages you download from CRAN, but these topics are outside the
scope of this book.

\begin{itemize}
\item
  \texttt{src/}: C, C++ and fortran source code
\item
  \texttt{exec/}: executable scripts
\item
  \texttt{po/}: translation files
\end{itemize}

\section{Getting started}

When creating a package the first thing (and sometimes the most
difficult) is to come up with a name for it. There's only one formal
requirement:

\begin{itemize}
\itemsep1pt\parskip0pt\parsep0pt
\item
  The package name can only consist of letters and numbers, and must
  start with a letter.
\end{itemize}

But I have a few additional recommendations:

\begin{itemize}
\item
  Make the package name googleable, so that if you google the name you
  can easily find it. This makes it easy for potential users to find
  your package, and it's also useful for you, because it makes it easier
  to find out who is using it.
\item
  Avoid using both upper and lower case letters: they make the package
  name hard to type and hard to remember. For example, I can never
  remember if it's \texttt{Rgtk2} or \texttt{RGTK2} or \texttt{RGtk2}.
\end{itemize}

Some strategies I've used in the past to create packages names:

\begin{itemize}
\item
  Use abbreviations: \texttt{lvplot} (letter value plots),
  \texttt{meifly} (models explored interactively)
\item
  Add an extra R: \texttt{stringr} (string processing), \texttt{tourr}
  (grand tours), \texttt{httr} (HTTP requests), \texttt{helpr}
  (alternative documentation view)
\item
  Find a name evocative of the problem and modify it so that it's
  unique: \texttt{plyr} (generalisation of apply tools),
  \texttt{lubridate} (makes dates and times easier), \texttt{mutatr}
  (mutable objects), \texttt{classifly} (high-dimensional views of
  classification)
\end{itemize}

Once you have a name, create a directory with that name, and inside that
create an \texttt{R} subdirectory and a \texttt{DESCRIPTION} file (note
that there's no extension, and the file name must be all upper case).

\section{The \texttt{R/} directory}

The \texttt{R/} directory contains all your R code, so copy in any
existing code.

It's up to you how you arrange your functions into files. There are two
possible extremes: all functions in one file, and each function in its
own file. I think these are both too extreme, and I suggest grouping
related functions into a single file. My rule of thumb is that if I
can't remember which file a function lives in, I probably need to split
them up into more files: having only one function in a file is perfectly
reasonable, particularly if the functions are large or have a lot of
documentation. As you'll see in the next chapter, often the code for the
function is small compared to its documentation (it's much easier to do
something than it is to explain to someone else how to do it.)

The next step is to create a \texttt{DESCRIPTION} file that defines
package metadata.

\section{A minimal \texttt{DESCRIPTION} file}

A minimal description file (this one is taken from an early version of
plyr) looks like this:

\begin{verbatim}
Package: plyr
Title: Tools for splitting, applying and combining data
Description: 
Version: 0.1
Author: Hadley Wickham <h.wickham@gmail.com>
Maintainer: Hadley Wickham <h.wickham@gmail.com>
License: MIT
\end{verbatim}

This is the critical subset of package metadata: what it's called
(\texttt{Package}), what it does (\texttt{Title}, \texttt{Description}),
who's allowed to use and distribute it (\texttt{License}), who wrote it
(\texttt{Author}), and who to contact if you have problems
(\texttt{Maintainer}). Here I've left the \texttt{Description} blank to
illustrate that if you haven't decided what the correct value is yet,
it's ok to leave it blank.

Again, the six required elements are:

\begin{itemize}
\item
  \texttt{Package}: name of the package. Should be the same as the
  directory name.
\item
  \texttt{Title}: a one line description of the package.
\item
  \texttt{Description}: a more detailed paragraph-length description.
\item
  \texttt{Version}: the version number, which should be of the the form
  \texttt{major.minor.patchlevel}. See \texttt{?package\_version} for
  more details on the package version formats. I recommended following
  the principles of \href{http://semver.org/}{semantic versioning}.
\item
  \texttt{Maintainer}: a single name and email address for the person
  responsible for package maintenance.
\item
  \texttt{License}: a standard abbreviation for an open source license,
  like \texttt{GPL-2} or \texttt{BSD}. A complete list of possibilities
  can be found by running
  \texttt{file.show(file.path(R.home(), "share/licenses/license.db"))}.
  If you are using a non-standard license, put \texttt{file LICENSE} and
  then include the full text of the license in a \texttt{LICENSE}.
\end{itemize}

\section{Other \texttt{DESCRIPTION} components}

A more complete \texttt{DESCRIPTION} (this one from a more recent
version of \texttt{plyr}) looks like this:

\begin{verbatim}
Package: plyr
Title: Tools for splitting, applying and combining data
Description: plyr is a set of tools that solves a common set of
    problems: you need to break a big problem down into manageable
    pieces, operate on each pieces and then put all the pieces back
    together.  For example, you might want to fit a model to each
    spatial location or time point in your study, summarise data by
    panels or collapse high-dimensional arrays to simpler summary
    statistics. The development of plyr has been generously supported
    by BD (Becton Dickinson).
URL: http://had.co.nz/plyr
Version: 1.3
Maintainer: Hadley Wickham <h.wickham@gmail.com>
Author: Hadley Wickham <h.wickham@gmail.com>
Depends: R (>= 2.11.0)
Suggests: abind, testthat (>= 0.2), tcltk, foreach
Imports: itertools, iterators
License: MIT
\end{verbatim}

This \texttt{DESCRIPTION} includes other components that are optional,
but still important:

\begin{itemize}
\item
  \texttt{Depends}, \texttt{Suggests}, \texttt{Imports} and
  \texttt{Enhances} describe which packages this package needs. They are
  described in more detail in {[}{[}namespaces{]}{]}.
\item
  \texttt{URL}: a url to the package website. Multiple urls can be
  separated with a comma or whitespace.
\end{itemize}

Instead of \texttt{Maintainer} and \texttt{Author}, you can
\texttt{Authors@R}, which takes a vector of \texttt{person()} elements.
Each person object specifies the name of the person and their role in
creating the package:

\begin{itemize}
\item
  \texttt{aut}: full authors who have contributed much to the package
\item
  \texttt{ctb}: people who have made smaller contributions, like
  patches.
\item
  \texttt{cre}: the package creator/maintainer, the person you should
  bother if you have problems
\end{itemize}

Other roles are listed in the help for person. Using \texttt{Authors@R}
is useful when your package gets bigger and you have multiple
contributors that you want to acknowledge appropriately. The equivalent
\texttt{Authors@R} syntax for plyr would be:

\begin{verbatim}
  Authors@R: person("Hadley", "Wickham", role = c("aut", "cre"))
\end{verbatim}

There are a number of other less commonly used fields like
\texttt{BugReports}, \texttt{KeepSource}, \texttt{OS\_type} and
\texttt{Language}. A complete list of the \texttt{DESCRIPTION} fields
that R understands can be found in the
\href{http://cran.r-project.org/doc/manuals/R-exts.html\#The-DESCRIPTION-file}{R
extensions manual}.

\section{Source, binary and bundled packages}

So far we've just described the structure of a source package: the
development version of a package that lives on your computer. There are
also two other types of package: bundled packages and binary packages.

A package \textbf{bundle} is a compressed version of a package in a
single file. By convention, package bundles in R use the extension
\texttt{.tar.gz}. This is Linux convention indicating multiple files
have been collapsed into a single file (\texttt{.tar}) and then
compressed using gzip (\texttt{.gz}). The package bundle is useful if
you want to manually distribute your package to another R package
developer. It is not OS specific. You can use \texttt{devtools::build()}
to make a package bundle.

If you want to distribute your package to another R user (i.e.~someone
who doesn't necessarily have the development tools installed) you need
to make a \textbf{binary} package. Like a package bundle, a binary
package is a single file, but if you uncompress it, you'll see that the
internal structure is a little different to a source package:

\begin{itemize}
\item
  a \texttt{Meta/} directory contains a number of \texttt{Rds} files.
  These contain cached metadata about the package, like what topics the
  help files cover and parsed versions of the \texttt{DESCRIPTION}
  files. (If you want to look at what's in these files you can use
  \texttt{readRDS})
\item
  a \texttt{html/} directory contains some files needed for the HTML
  help.
\item
  there are no \texttt{.R} files in the \texttt{R/} directory - instead
  there are three files that store the parsed functions in an efficient
  format. This is basically the result of loading all the R code and
  then saving the functions with \texttt{save}, but with a little extra
  metadata to make things as fast as possible.
\item
  If you had any code in the \texttt{src/} directory there will now be a
  \texttt{libs/} directory that contains the results of compiling that
  code for 32 bit (\texttt{i386/}) and 64 bit (\texttt{x64})
\end{itemize}

Binary packages are platform specific: you can't install a Windows
binary package on a Mac or vice versa. You can use
\texttt{devtools::build(binary = TRUE)} to make a package bundle.

An \textbf{installed} package is just a binary package that's been
uncompressed into a package library, described next.

\section{Package libraries}

A library is a collection of installed packages. You can have multiple
libraries on your computer and most people have at least two: one for
the recommended packages that come with a base R install (like
\texttt{base}, \texttt{stats} etc), and one library where the packages
you've installed live. The default is to make that directory dependent
on which version of R you have installed - that's why you normally lose
all your packages when you reinstall R. If you want to avoid this
behaviour, you can manually set the \texttt{R\_LIBS} environmental
variable to point somewhere else. \texttt{.libPaths()} tells you where
your current libraries live.

When you use \texttt{library(pkg)} to load a package, R looks through
each path in \texttt{.libPaths()} to see if a directory called
\texttt{pkg} exists.

\section{Installing packages}

Package installation is the process whereby a source package gets
converted into a binary package and then installed into your local
package library. There are a number of tools that automate this process:

\begin{itemize}
\item
  \texttt{install.packages()} installs a package from CRAN. Here CRAN
  takes care of making the binary package and so installation from CRAN
  basically is equivalent to downloading the binary package value and
  unzipping it in \texttt{.libPaths(){[}1{]}} (but you should never do
  this by hand because the process also does other checks)
\item
  \texttt{devtools::install()} installs a source package from a
  directory on your computer.
\item
  \texttt{devtools::install\_github()} installs a package that someone
  has published on their \href{http://github}{github} account. There are
  a number of similar functions that make it easy to install packages
  from other internet locations: \texttt{install\_url},
  \texttt{install\_gitorious}, \texttt{install\_bitbucket}, and so on.
\end{itemize}

\section{Exercises}

(to be integrated throughout the chapter)

\begin{itemize}
\item
  Go to CRAN and download the source and binary for XXX. Unzip and
  compare. How do they differ?
\item
  Download the \textbf{source} packages for XXX, YYY, ZZZ. What
  directories do they contain?
\item
  Where is your default library? What happens if you install a new
  package from CRAN?
\end{itemize}
