\hypertarget{how-to-write-a-reproducible-example}{%
\chapter{How to write a reproducible
example}\label{how-to-write-a-reproducible-example}}

You are most likely to get good help with your R problem if you provide
a reproducible example. A reproducible example allows someone else to
recreate your problem by just copying and pasting R code.

There are four things you need to include to make your example
reproducible: required packages, data, code, and a description of your R
environment.

\begin{itemize}
\item
  \textbf{Packages} should be loaded at the top of the script, so it's
  easy to see which ones the example needs.
\item
  The easiest way to include \textbf{data} in an email is to use dput()
  to generate the R code to recreate it. For example, to recreate the
  mtcars dataset in R, I'd perform the following steps:

  \begin{enumerate}
  \def\labelenumi{\arabic{enumi}.}
  \tightlist
  \item
    Run \texttt{dput(mtcars)} in R
  \item
    Copy the output
  \item
    In my reproducible script, type \texttt{mtcars\ \textless{}-} then
    paste.
  \end{enumerate}
\item
  Spend a little bit of time ensuring that your \textbf{code} is easy
  for others to read:

  \begin{itemize}
  \item
    make sure you've used spaces and your variable names are concise,
    but informative
  \item
    use comments to indicate where your problem lies
  \item
    do your best to remove everything that is not related to the
    problem. The shorter your code is, the easier it is to understand.
  \end{itemize}
\item
  Include the output of sessionInfo() as a comment. This summarises your
  \textbf{R environment} and makes it easy to check if you're using an
  out-of-date package.
\end{itemize}

You can check you have actually made a reproducible example by starting
up a fresh R session and pasting your script in.

Before putting all of your code in an email, consider putting it on
http://gist.github.com/. It will give your code nice syntax
highlighting, and you don't have to worry about anything getting mangled
by the email system.

\hypertarget{example}{%
\section{Example}\label{example}}

Here's an illustration of how to create a reproducible example. First,
have R print out your data in a format that can be copy-pasted:

\begin{Shaded}
\begin{Highlighting}[]
\CommentTok{# For this example, use the built-in BOD data set. Replace this with your data.}
\KeywordTok{dput}\NormalTok{(BOD)}
\end{Highlighting}
\end{Shaded}

\begin{verbatim}
## structure(list(Time = c(1, 2, 3, 4, 5, 7), demand = c(8.3, 10.3, 
## 19, 16, 15.6, 19.8)), class = "data.frame", row.names = c(NA, 
## -6L), reference = "A1.4, p. 270")
\end{verbatim}

Then you can use that output to create a reproducible example:

\begin{Shaded}
\begin{Highlighting}[]
\KeywordTok{library}\NormalTok{(ggplot2)}

\CommentTok{# Save the data structure in variable BOD}
\NormalTok{BOD <-}\StringTok{ }\KeywordTok{structure}\NormalTok{(}\KeywordTok{list}\NormalTok{(}\DataTypeTok{Time =} \KeywordTok{c}\NormalTok{(}\DecValTok{1}\NormalTok{, }\DecValTok{2}\NormalTok{, }\DecValTok{3}\NormalTok{, }\DecValTok{4}\NormalTok{, }\DecValTok{5}\NormalTok{, }\DecValTok{7}\NormalTok{), }\DataTypeTok{demand =} \KeywordTok{c}\NormalTok{(}\FloatTok{8.3}\NormalTok{, }\FloatTok{10.3}\NormalTok{,}
\DecValTok{19}\NormalTok{, }\DecValTok{16}\NormalTok{, }\FloatTok{15.6}\NormalTok{, }\FloatTok{19.8}\NormalTok{)), }\DataTypeTok{.Names =} \KeywordTok{c}\NormalTok{(}\StringTok{"Time"}\NormalTok{, }\StringTok{"demand"}\NormalTok{), }\DataTypeTok{row.names =} \KeywordTok{c}\NormalTok{(}\OtherTok{NA}\NormalTok{,}
\OperatorTok{-}\NormalTok{6L), }\DataTypeTok{class =} \StringTok{"data.frame"}\NormalTok{, }\DataTypeTok{reference =} \StringTok{"A1.4, p. 270"}\NormalTok{)}

\CommentTok{# Some example code that uses the data}
\KeywordTok{ggplot}\NormalTok{(BOD, }\KeywordTok{aes}\NormalTok{(}\DataTypeTok{x=}\NormalTok{Time, }\DataTypeTok{y=}\NormalTok{demand)) }\OperatorTok{+}\StringTok{ }\KeywordTok{geom_line}\NormalTok{()}
\end{Highlighting}
\end{Shaded}

Check that others can run this code by simply copying and pasting it in
a \textbf{new} R sesion.
