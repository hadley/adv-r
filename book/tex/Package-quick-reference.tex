\chapter{Quick reference sheet for package process}

In progress

All devtools package functions take a package path as the first
argument. If it is omitted it assumes the package is in the current
working directory - developing the package in this way is best practice.

\section{Create a new package}

\begin{itemize}
\item
  With no R files: \texttt{create("mypackage")}
\item
  With R files: copy in to a directory which contains a DESCRIPTION file
  and a directory called \texttt{R/} and then run \texttt{load\_all()}
\end{itemize}

\section{Development workflow}

\begin{itemize}
\item
  Re-loading the package during development: \texttt{load\_all()} (this
  loads dependent packages, data, R code and compiled src code)
\item
  Converting roxygen comments to Rd files: \texttt{document()}
\item
  Check that in-development documentation looks ok:
  \texttt{dev\_help("topic")}
\end{itemize}

\section{Deploying}

\begin{itemize}
\itemsep1pt\parskip0pt\parsep0pt
\item
  Locally
\item
  On github
\item
  On CRAN
\end{itemize}

\subsection{Building the package for all users of the system}

Change to the path of the package

\begin{Shaded}
\begin{Highlighting}[]
\KeywordTok{document}\NormalTok{()}
\KeywordTok{install}\NormalTok{()}
\end{Highlighting}
\end{Shaded}

Check the error messages!
