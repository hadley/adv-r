\hypertarget{formulas}{%
\section{Formulas}\label{formulas}}

There is one other important tool of non-standard evaluation: the
formula. The formula operator, \texttt{\textasciitilde{}}, is used
extensively by modelling functions, but also by some graphics functions
(e.g.~lattice and \texttt{plot}) and a few data manipulation functions
(e.g. \texttt{xtabs()} and \texttt{aggregate()}).

\hypertarget{formula-as-a-quoting-function}{%
\section{Formula as a quoting
function}\label{formula-as-a-quoting-function}}

There are two advantages for using \texttt{\textasciitilde{}} over
\texttt{quote()}:

\begin{itemize}
\tightlist
\item
  It is shorter
\item
  It captures both the expression and the environment in which it was
  evaluated
\end{itemize}

The disadvantage of using \texttt{\textasciitilde{}} is that most people
are used to its role in models, and may be surprised if the semantics
you imply from it are substantially different from standard modelling
formulas.

The formula object is a call that knows in which environment it was
evaluated. You can use \texttt{length()} to determine if it is one-sided
or two-sided, and \texttt{{[}{[}} to extract the various pieces.

\begin{Shaded}
\begin{Highlighting}[]
\NormalTok{f1 <-}\StringTok{ }\ErrorTok{~}\StringTok{ }\NormalTok{a }\OperatorTok{+}\StringTok{ }\NormalTok{b}
\KeywordTok{length}\NormalTok{(f1)}
\NormalTok{f1[[}\DecValTok{1}\NormalTok{]]}
\NormalTok{f1[[}\DecValTok{2}\NormalTok{]]}

\NormalTok{f2 <-}\StringTok{ }\NormalTok{y }\OperatorTok{~}\StringTok{ }\NormalTok{a }\OperatorTok{+}\StringTok{ }\NormalTok{b}
\KeywordTok{length}\NormalTok{(f2)}
\NormalTok{f2[[}\DecValTok{1}\NormalTok{]]}
\NormalTok{f2[[}\DecValTok{2}\NormalTok{]]}
\NormalTok{f2[[}\DecValTok{3}\NormalTok{]]}
\end{Highlighting}
\end{Shaded}

You can extract the environment of a formula with
\texttt{environment()}, as demonstrated with this implementation of
\texttt{subset()}:

\begin{Shaded}
\begin{Highlighting}[]
\NormalTok{subset_f <-}\StringTok{ }\ControlFlowTok{function}\NormalTok{(x, f) \{}
  \KeywordTok{stopifnot}\NormalTok{(}\KeywordTok{inherits}\NormalTok{(f, }\StringTok{"formula"}\NormalTok{), }\KeywordTok{length}\NormalTok{(f) }\OperatorTok{==}\StringTok{ }\DecValTok{2}\NormalTok{)}
\NormalTok{  r <-}\StringTok{ }\KeywordTok{eval}\NormalTok{(f[[}\DecValTok{2}\NormalTok{]], x, }\KeywordTok{environment}\NormalTok{(f))}
\NormalTok{  x[r, ]}
\NormalTok{\}}
\KeywordTok{subset_f}\NormalTok{(mtcars, }\OperatorTok{~}\StringTok{ }\NormalTok{cyl }\OperatorTok{==}\StringTok{ }\DecValTok{4}\NormalTok{)}
\end{Highlighting}
\end{Shaded}

Note that because the code is evaluated in the environment associated
with the formula, the semantics are a little different if you're
creating the formula in a function:

\begin{Shaded}
\begin{Highlighting}[]
\NormalTok{f <-}\StringTok{ }\ControlFlowTok{function}\NormalTok{(x) }\OperatorTok{~}\StringTok{ }\NormalTok{cyl }\OperatorTok{==}\StringTok{ }\NormalTok{x}
\KeywordTok{subset_f}\NormalTok{(mtcars, }\KeywordTok{f}\NormalTok{(}\DecValTok{4}\NormalTok{))}
\end{Highlighting}
\end{Shaded}

\hypertarget{xtabs}{%
\subsection{\texorpdfstring{\texttt{xtabs()}}{xtabs()}}\label{xtabs}}

Is a pretty horrible example because of it's combination of call
mangling and tangles with sparse matrices.

\hypertarget{formulas-for-modelling}{%
\section{Formulas for modelling}\label{formulas-for-modelling}}

Keep it brief: focus on main concepts (possibly showing complete lm
implmentation using Rcpp), and pointing to documentation where
necessary. Need to discuss specials (e.g.~offset/Error) and how splines
work?

White book.

Patsy: http://patsy.readthedocs.org/en/latest/R-comparison.html

Formula package
(http://cran.r-project.org/web/packages/Formula/vignettes/Formula.pdf)

Models use two steps: first converting the formula into matrices, and
then manipulating using matrix algebra.

\begin{itemize}
\item
  RcppEigen:::fastLm.formula
\item
  http://developer.r-project.org/model-fitting-functions.txt
\item
  terms, terms.object, terms.formula
\item
  model.response, model.weights
\item
  model.matrix, model.frame
\item
  lm.fit
\end{itemize}
